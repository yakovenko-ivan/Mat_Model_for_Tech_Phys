% Latex шаблон домашнего задания по курсу Матеметическое моделирование в технической физике
% для групп магистратуры кафедры "Физика" 
% факультета "Фундаментальные науки" МГТУ им. Н.Э. Баумана.
% разработано доцентом кафедры "Физика" Яковенко И.С. на основе шаблона
% Diogo Correia and João Santos
% Repository: https://github.com/dvcorreia/engenius-ua-latex-template.git

\documentclass[FN]{math_mod}

% Параметры титульного листа
% Название работы #1 (общее)
\title{Домашнее задание №1}
% Название работы #2 (более конкретно)
\titletwo{Краткое руководство по использованию \LaTeX~шаблона math\_mod.cls}

% Номер группы
\groupnumber{23}

% Вариант
\variant{1}

% ФИО Автора и контактные данные
\author{Фамилия И.О. \\ \email{emailauthor2@ua.pt}  \and Фамилия И.О. \\ \email{emailauthor2@ua.pt} }

% Время компиляции документа
\date{\today}

% Версия документа (1я сдача, 2я сдача с учетом замечаний, 3я сдача и т.д.)
\version{1.0}

\begin{document}

\maketitle

\begin{abstract}

В \textbf{аннотации} приводится краткая информация о целях работы, постановке задачи и и результатах. Не более 5 строк.  

\end{abstract}

\section{Общая информация}
Этот простой шаблон разработан с целью использования пользователем без глубоких навыков работы с \TeX~или \LaTeX. Для компиляции можно использовать как локальные инструменты, такие как \href{https://www.texstudio.org/}{TexStudio}, так и онлайн ресурсы, к примеру \href{https://www.overleaf.com/}{Overleaf}. В \textbf{Overleaf} данный шаблон и класс \textbf{math\_mod.cls} не требуют каких-либо дополнительных настроек, в TeXstudio необходимо установить пакет \href{https://www.ctan.org/pkg/minted}{Minted} для подсветки синтаксиса и настроить компиляцию с флагом \textbf{-shell-escape} (см. к примеру руководство на \href{https://tex.stackexchange.com/questions/99475/how-to-invoke-latex-with-the-shell-escape-flag-in-texstudio-former-texmakerx}{Stack Exchange}). Для знакомства с основными командами \LaTeX~рекомендуются материалы:
\begin{itemize}
    \item Небольшой справочник по \LaTeX~от команды Overleaf   (\href{https://ru.overleaf.com/learn/latex/Main_Page}{link})
    \item Инструкции по офомлению научных статей в \LaTeX~от издателя Elsevier. Имеется шаблон и руководство по работе с ним (\href{https://www.elsevier.com/authors/author-schemas/latex-instructions}{link}) 
    \item Краткое руководство от Евгения Балдина  (\href{https://www.ibm.com/developerworks/ru/library/latex_tutorial_01/index.html}{link}) 
    \item Wiki-подобная энциклопедия по \LaTeX~ (\href{https://ru.wikibooks.org/wiki/LaTeX}{link}) 
    \item он-лайн курс от ВШЭ на Coursera.org (\href{https://www.coursera.org/learn/latex}{link}) 
\end{itemize}

\section{Заголовок}
Перед тем как оформлять содержательную часть работы, необходимо оформить заголовок и титульный лист. В частности обозначить заголовок №1, (\mintinline{latex}{\title{}}) в котором указывается номер домашнего задания, и заголовок №2, (\mintinline{latex}{\titletwo{}}) в котором указывается конкретно, чему посвящена работа. Далее в полях \mintinline{latex}{\groupnumber{}} и \mintinline{latex}{\variant{}} указываются номер группы и вариант соответственно. 
Далее, необходимо указать авторов работы (\mintinline{latex}{\author}). Если автор один, информацию о втором авторе можно удалить или закомментировать, для комментарев в \LaTeX~ используется символ \%. Наконец, нужно заполнить поле 
\mintinline{latex}{\version{}}, где указывается версия работы. Скажем при первой сдаче пишется номер 1.0, далее если в работе обнаружены ошибки и она передана назад на доработку, после доработки пишется номер 2.0 и т.д. 

После общей информации заполняется краткая аннотация работы в блоке между \mintinline{latex}{\begin{abstract}} и \mintinline{latex}{\end{abstract}}. Аннотация представляет собой краткое содержание проекта с описанием целей работы и полученных результатов.  

\section{Содержание работы}

Далее приведем основные сведения по основным командам \LaTeX~для оформления работы. 

\subsection{Заголовки}
\label{section}

Заголовки, подзаголовки и подподзаголовки оформляются в \LaTeX~командами \mintinline{latex}{\section{}}, \mintinline{latex}{\subsection{}}, \mintinline{latex}{\subsubsection{}}. К каждому заголовку можно добавить ярлык командой \mintinline{latex}{\label{}}, и в последствии ссылаться на него в тексте командой \mintinline{latex}{\ref{}}. Пример:

Ссылка на параграф ``Заголовки''~\ref{section}.

\subsection{Уравнения}

Уравнения можно писать как в тексте, с выделяя их символами \$: \mintinline{latex}{$x+y=c$}, или в отдельной строке, в блоке между \mintinline{latex}{\begin{equation}} и \mintinline{latex}{\end{equation}}:  

\begin{codebox}{Код для генерации уравнений}
    \begin{minted}{Latex}
        \begin{equation}
            F(x) = \int^a_b \frac{1}{3}x^3 dx
            \label{eq:func}
        \end{equation}
    \end{minted}
\end{codebox}

Результат:
\begin{equation}
    F(x) = \int^a_b \frac{1}{3}x^3 dx
    \label{eq:func}
\end{equation}

Здесь также можно добавить ярлык \mintinline{latex}{\label{}}, и сослаться на это Уравнение~\ref{eq:func}. Если на уравнение в тексте не ссылаются и номер для уравнения не требуется, используется блок \mintinline{latex}{\begin{equation*}} и \mintinline{latex}{\end{equation*}}.

Примеры кода и сгенерированных уравнений можно также найти в Jupyter блокнотах лекций. 

\subsection{Списки}

Списки генерируются с использоваением блока \mintinline{latex}{\begin{itemize}} и \mintinline{latex}{\end{itemize}}, с их помощью можно делать вложенные списки:

\begin{codebox}{Код для генерации списков}
    \begin{minted}{Latex}
		\begin{itemize}
			\item item 1
			\item group 1
			\begin{itemize}
				\item item 2
				\item item 3
				\begin{itemize}
    				\item item 4
    				\item item 5
      			\end{itemize}
			\end{itemize}
			\item item 6
			\item item 7
		\end{itemize}
    \end{minted}
\end{codebox}

Результат:

\begin{itemize}
	\item item 1
	\item group 1
	\begin{itemize}
		\item item 2
		\item item 3
		\begin{itemize}
			\item item 4
			\item item 5
  		\end{itemize}
	\end{itemize}
	\item item 6
	\item item 7
\end{itemize}

\subsection{Images}

Картинки вставляются в текст с помощью блока \mintinline{latex}{\begin{figure}} и \mintinline{latex}{\end{figure}}. Здесь также можно добавлять ссылки с помощью \mintinline{latex}{\label{eq:func}}. В блоке \mintinline{latex}{\begin{figure}} для загрузки картинки используется команда \mintinline{latex}{\includegraphics[width=ширина картинки]{путь к картинке}}. Ширина задается в определенных единицах (cm, in, em ...). Подпись к рисунку задается командой \mintinline{latex}{\caption}.

Пример кода загружающего картинку:

\begin{codebox}{Код для вставки рисунков}
    \begin{minted}{Latex}
        \begin{figure}[H]
            \begin{center}
                \includegraphics[width=10cm]{src/BMSTU.png}
                \caption{Рисунок 1. Герб МГТУ.}
                \label{img:bmstu_logo}
            \end{center}
        \end{figure}
    \end{minted}
\end{codebox}

\begin{figure}[H]
    \begin{center}
        \includegraphics[width=\textwidth/3]{src/BMSTU.png}
        \caption{Рисунок 1. Герб МГТУ.}
        \label{img:bmstu_logo}
    \end{center}
\end{figure}

\subsection{Code}

Программный код можно вставлять как в текст черезе \mintinline{Latex}{\mintinline}: \mintinline{python}{print("Hello world!")}, так и в отдельные блоки с помощью блока \mintinline{Latex}{\begin{codebox}} и \mintinline{Latex}{\end{codebox}}. Внутри блока используется второй блок \mintinline{Latex}{\begin{minted}{Язык программирования}} и \mintinline{Latex}{\end{minted}}. Пример блока с кодом:

\begin{codebox}{Пример кода Python}
    \begin{minted}{python}
        # Функция для нахождения n-го числа Фибоначчи. 
          
        def Fibonacci(n): 
            if n<0: 
                print("Incorrect input") 
            # First Fibonacci number is 0 
            elif n==1: 
                return 0
            # Second Fibonacci number is 1 
            elif n==2: 
                return 1
            else: 
                return Fibonacci(n-1)+Fibonacci(n-2) 
          
        # Запуск функции 
          
        print(Fibonacci(9)) 
          
        #Код отсюда (https://www.geeksforgeeks.org/python-program-for-program-for-fibonacci-numbers-2/) 

    \end{minted}
\end{codebox}

\subsection{Условие задачи}
Условия задач домашнего задания выписываются в отдельном блоке \mintinline{Latex}{\begin{problem}{№ задачи}} и \mintinline{Latex}{\end{problem}}. Пример

\begin{codebox}{Код для вставки рисунков}
    \begin{minted}{Latex}
        \begin{problem}{1}
        Реализуйте схему "КАБАРЕ" для решения уравнения Хопфа:
        \begin{equation}
            \frac{\partial u}{\partial t} = - u \frac{\partial u}{\partial x}
        \end{equation}
        Определите область устойчивости схемы, проанализируйте её диссипативные и дисперсионные характеристики.
        \end{problem}
    \end{minted}
\end{codebox}

\begin{problem}{1}
Реализуйте схему "КАБАРЕ" для решения уравнения Хопфа:
\begin{equation}
    \frac{\partial u}{\partial t} = - u \frac{\partial u}{\partial x}
\end{equation}
Определите область устойчивости схемы, проанализируйте её диссипативные и дисперсионные характеристики.
\end{problem}

\subsection{Ссылки}

Особенно просто с помощью \LaTeX~генерируются списки литературы. Если в работе используются или делается обзор на результаты сторонних авторов, всегда необходимо ссылаться на соответствующие работы. Библиография записывается в отдельный файл \mintinline{text}{./biblio.bib}. В качестве примера сейчас там приведены три работы. Сослаться на них можно используя команду \mintinline{Latex}{\cite{название работы}}: Работа Эйнштейна по электродинамике движущихся тел \cite{einstein}. Книга Frank Mittelbach с соавторами ``The LaTeX Companion (Tools and Techniques for Computer Typesetting)'' \cite{latexcompanion}. Веб-сайт создателя \TeX~Дональда Кнута \cite{knuthwebsite}. Как только в тексте появляется ссылка на работу, она автоматически помещается в список литературы в конце документа.  

\section{Вопросы, ошибки и предложения}

Вопросы по использованию класса math\_mod.cls, вопросы по данному примеру использования этого класса, ошибки и предложения по их модификации пишите в разделе Issues на \href{https://github.com/yakovenko-ivan/Mat_Model_for_Tech_Phys}{сайте} github репозитория курса ``Математическое моделирование в технической физике''. Конекретные модификации можно предлагать и виде Pull Requests, они будут обязательно рассмотрены.  

\bibliography{biblio}

\end{document}
